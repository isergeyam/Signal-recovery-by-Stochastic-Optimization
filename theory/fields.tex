Непрерывное векторное поле $f: \bR^m \to \bR^m$ называется \emph{строго монотонным} с модулем монотонности $\varkappa$ на множестве $Z \subset \mathbf{R}^m$, если для всех $z, z' \in Z$ выполняется:
\begin{equation}
    \label{mon_mod}
    \left[g(z) - g(z')\right]^T\left[z - z'\right] \geqslant \varkappa \|z-z'\|_2^2
\end{equation}

Будем говорить, что поле $f$ --- \emph{строго монотонное} на $Z$, если его модуль монотонности положителен.

Для дальнейшей работы потребуются следующие свойства монотонных векторных полей:
\begin{enumerate}
    \item [\textbf{I.}] Для монотонного векторного поля $f$ в пространстве $\bR^m$ и $n \times m$-матрицы $A$ векторное поле
    \begin{equation*}
        g(x) = Af(A^Tx + a)
    \end{equation*}
    является монотонным полем в пространстве $\bR^n$. Кроме того, если $f$ монотонно с модулем монотонности $\varkappa \geqslant 0$ на замкнутом
    выпуклом множестве $Z \subset \bR^m$, а замкнутое выпуклое множество $X \subset \bR^n$ таково, что 
    $A^T x + a \in Z$ при $x\in X$, то на множестве $X$ поле $g$ монотонно с модулем монотонности $\sigma^2\varkappa$, где $\sigma$ --- минимальное
    сингулярное число матрицы $A$.
    
    \item [\textbf{II.}] Если $S$ --- польское пространство и $f(x, s): \bR^m \times S \to \bR^m$ --- борелевская функция, являющаяся монотонным полем при любом фиксированном $s \in S$, а $\mu(ds)$ --- борелевская вероятностная мера на $S$, а векторное поле
    \begin{equation}
        \label{vec_field_summ}
        F(x) = \int_S f(x,s) \mu(ds)
    \end{equation}
    корректно определено при всех $x$, то поле $F$ монотонно. Кроме того, если $X$ --- замкнутое выпуклое подмножество $\bR^m$, а борелевская функция $\varkappa(s)$ такова, что при всех $s\in S$ поле $f(\cdot, s)$ монотонно на $X$ с модулем монотонности $\varkappa(s)$, то поле $F$ также монотонна на $X$, с модулем монотонности $\int_S \varkappa(s) \mu(ds)$.
    
\end{enumerate}
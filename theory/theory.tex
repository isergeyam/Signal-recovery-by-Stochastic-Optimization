%\usepackage{multirow}
\documentclass[11pt]{article}
\usepackage{amsmath}
\usepackage{color}
\usepackage[colorlinks=true]{hyperref}
\usepackage{url}
\usepackage[russian]{babel}
\usepackage[utf8]{inputenc}
\selectlanguage{russian}


\def\uphi{\varphi}


\def\r{{\hbox{\tiny\rm ,r}}}
\def\l{{\hbox{\tiny\rm ,l}}}
\def\up{\hbox{\tt red}}
\def\dw{\hbox{\tt blue}}
\def\mysix{8}
\def\LwB{{\hbox{\rm LwB}}}
\def\Riskopt{\Risk_{\hbox{\tiny\rm opt}}}
\def\RiskS{{\hbox{\rm RiskS}}}
\def\RiskoptS{\hbox{\rm RiskS}_{\hbox{\tiny\rm opt}}}
\def\tRiskoptS{\hbox{\rm \scriptsize RiskS}_{\hbox{\tiny\rm opt}}}
\def\Size{{\hbox{\rm Size}}}
\def\mes{{\hbox{\rm mes}}}
\def\cl{\mathop{\hbox{\rm cl}}}
\def\inter{\mathop{\hbox{\rm int}}}
\def\supp{\mathop{\hbox{\rm supp}}}
\def\cond{\mathop{\hbox{\rm\small Cond}}}
\def\I{{\cal I}}
\def\tDiag{\hbox{\rm\scriptsize Diag}}
\def\brA{{{A}}}
\def\brB{{{B}}}
\def\cX{{\cal X}}
\def\brX{{\Y}}
\def\brn{{\bar{n}}}
\def\brF{{\bar{F}}}
\def\bT{{\mathbf{T}}}
\def\bK{{\mathbf{K}}}
\newcommand{\half}{ \mbox{\small$\frac{1}{2}$}}
\newcommand{\four}{\mbox{\small$\frac{1}{4}$}}
\usepackage{amsfonts}
\usepackage{epsfig}

\def\Risk{{\hbox{\rm Risk}}}
\def\Ker{\mathop{\hbox{\rm Ker}}}
\def\cD{{\cal D}}
\def\cZ{{\cal Z}}
\def\cO{{\cal O}}
\def\Erf{{\mathop{\hbox{\small\rm Erf}}}}
\def\ErfInv{{\mathop{\hbox{\small\rm ErfInv}}}}
\def\tErfInv{{\mathop{\hbox{\rm\tiny ErfInv}}}}
\usepackage{amssymb}
\def\SadVal{\mathop{\hbox{\rm SadVal}}}
\def\cR{{\cal R}}
\def\Diag{{\hbox{\rm Diag}}}
\usepackage{graphicx}

\oddsidemargin=0truecm
\topmargin=-1truecm
\textwidth=16.0truecm
\textheight=23.0truecm
\def\bR{{\mathbf{R}}}
\def\bZ{{\mathbf{Z}}}
\def\Opt{{\hbox{\rm Opt}}}
\def\Ext{\hbox{\rm Ext}}
\def\cA{{\cal A}}
\def\cL{{\cal L}}
\def\cU{{\cal U}}
\def\cSG{{\cal SG}}
\def\cT{{\cal T}}
\def\cE{{\cal E}}
\def\cP{{\cal P}}
\def\cQ{{\cal Q}}
\def\cR{{\cal R}}
\def\cG{{\cal G}}
\def\cH{{\cal H}}
\def\cM{{\cal M}}
\def\cX{{\cal X}}
\def\cY{{\cal Y}}
\def\cN{{\cal N}}
\def\cC{{\cal C}}
\def\cS{{\cal S}}
\def\cK{{\cal K}}
\def\cF{{\cal F}}
\def\bE{{\mathbf{E}}}
\def\bS{{\mathbf{S}}}
\def\Prob{\hbox{\rm Prob}}
\def\Erf{\hbox{\rm Erf}}
\def\Argmin{\mathop{\hbox{\rm Argmin}}}
\def\ML{{\mathrm{\tiny ML}}}
\def\sML{{\textrm{\tiny ML}}}
\def\VI{{\textrm{VI}}}
\def\Proj{{\mathrm{Proj}}}
\def\argmin{\mathop\mathrm{argmin}}

\newtheorem{lemma}{Лемма}[section]
\newtheorem{corollary}{Следствие}[section]
\newtheorem{proposition}{Утверждение}[section]
\newtheorem{remark}{Замечание}[section]
\newtheorem{theorem}{Теорема}[section]
\newtheorem{example}{Пример}[section]
\newtheorem{definition}{Определение}[section]




\makeatletter
\def\@seccntformat#1{%
  \expandafter\ifx\csname c@#1\endcsname\c@section\else
  \csname the#1\endcsname\quad
  \fi}
\makeatother

\title{Восстановление сигналов методами стохастической оптимизации}
\author{Василевский Алексей, Григорянц Сергей, Потапов Георгий, Федорец Никита}
\date{Октябрь 2019}

\begin{document}

\maketitle

\section{Введение:}

Пусть $\omega^K=(\omega_1, ..., \omega_K)$, $\omega_k=(\eta_k, y_k)$, где $\eta\in \mathbf{R}^{n\times m}$, $y_k\in \mathbf{R}^{m}$
Предположим, что наблюдения описываются Generalized Linear Model, то есть условное математическое ожидание при условии $\eta$ по $y$ есть $f(\eta^Tx),\ f:R^m\rightarrow R^m$, где $x\in R^m$ -
неизвестный вектор параметров.
Цель - восстановить $x$ по наблюдениям $\omega^K$. Стандартный подход - выбрать функцию потерь и $l(y, \Theta):\mathbf{R}^m\times \mathbf{R}^m\rightarrow \mathbf{R}$ и восстановить $x$, как оптимальное решение оптимизационной задачи
\begin{eqnarray}
\min_{u\in\cal X}\mathbf{E}_{\omega\sim P_x}\{l(y,f(\eta^T u))\},
  \label{1}
\end{eqnarray}
где $P_x$ - распределение наблюдений $\omega$, а $\cal X$ заранее известное множество параметров. Другими словами, это сводится к задаче стохастической оптимизации (1), у которой неточное решение ищется через наблюдения $\omega^K$.
Это так же может быть сделано через усреднение
\begin{eqnarray}
  \frac{1}{K}\sum_{k=1}^{K}l(y_k,f(\eta_k^T u))
    \label{2}
  \end{eqnarray}
математического ожидания (1) по $x\in \cal{X}$ методом {\em Sample Average Approximation} (SAA),
или применив итеративный алгоритм стохастической оптимизации {\em Stochastic Approximation} (SA).\\
Предполагая, что условная вероятность при условии $\eta$ $y$ индуцированно $P_x$ принадлежит известному параметрическому семейству 
$\cP=\{P^\theta:\,
\theta\in \Theta\subset\bR^m\},$
в частности $P_{|\eta}^x=P^{f(\eta^Tx)}$, стандартный выбор функции потерь определяется с помощью максимального правдоподобия: при условии, что распределение $P$ имеет плотность $p_\theta$, тогда
$$
\ell(y,\theta)=-\ln(p_\theta(y)).
$$\\
Например, в классической линейной регрессии $m=1$, $f(s)=(1+e^{-s})^{-1}$, $\Theta =(0,1)$, и
$P^{\theta}$, $\theta \in \Theta$, это распределение Бернулли, то есть $y$ принимает значение 1 с
вероятностью $(1+\exp\{-\eta^T x\})^{-1}$ и 0 с соответствующей, тогда
$$
\ell(y,f(\eta^T u))=\ln(1+\exp\{\eta^T u\})-y\eta^T u.
$$
В этом случае, задача (1) становится задачей оптимизации
\begin{equation}\label{ait1}
\min_{u\in \cX} \bE_{(\eta,y)\sim P_x}\left\{\ln(1+\exp\{\eta^T u\})-y\eta^T u\right\},
\end{equation}
и ее SAA становится
\begin{equation}\label{ait3}
\min_{u\in \cX} {1\over K}\sum_{k=1}^K\left[\ln(1+\exp\{\eta_k^T u\})-y_k\eta_k^T u\right];
\end{equation}
Предполагая, что $\cX$ выпуклая, обе эти задачи становятся выпуклыми, что подразумевает возможность глобального решения SAA достаточно эффективно, аналогично использованию хороших свойств сходимости SA.\\
Более обще, распределение наблюдений из экспоненциального семейства, отрицательная логарифмическая функция правдоподобия имеет вид
\[
\{\ell(y,\eta^T u)=F(\eta^T u)-y\eta^T u,
\]
с выпуклой функцией распределения $F$, и соответствующей минимизацией функции потерь
$$
\min_{u\in \cX} \bE_{(\eta,y)\sim P_x}
\left\{F(\eta^Tu)-y\eta^Tu\right\}.
$$
В этом случае, так же как и в случае логистической регрессии, SAA или SA могут применяться для вычисления
Оценки максимального правдоподобия параметра модели.
Однако, что предположение на экспоненциальная семейство довольно сильное. Если 
$$
y=f(\eta^Tx)+\xi,\,\,\xi\sim\cN(0,\sigma^2I_m).
$$
 $\ell(\cdot)$ становится
$$
\min_{u\in \cX}\bE_{\eta\sim Q}\{\|f(\eta^T x)-f(\eta^T u)\|_2^2\},\newline
\min_{u\in \cX}\left\{{1\over K}\sum_{k=1}^k\|y_k-f(\eta_k^T u)\|_2^2\right\},
$$
где $Q$ распределение $\eta$. Если $f$ нелинейна, тогда обе задачи обычно не выпуклы. 
Цель следующего заключается в том, чтобы предложить альтернативу подгонке модели через (1) на основе ML
 подхода с использованием функции потерь для оценки параметров.

\subsection{Необходимые определения и факты:}
Непрерывное векторное поле $f: \bR^m \to \bR^m$ называется \emph{строго монотонным} с модулем монотонности $\varkappa$ на множестве $Z \subset \mathbf{R}^m$, если для всех $z, z' \in Z$ выполняется:
\begin{equation}
    \label{mon_mod}
    \left[g(z) - g(z')\right]^T\left[z - z'\right] \geqslant \varkappa \|z-z'\|_2^2
\end{equation}

Будем говорить, что поле $f$ --- \emph{строго монотонное} на $Z$, если его модуль монотонности положителен.

Для дальнейшей работы потребуются следующие свойства монотонных векторных полей:
\begin{enumerate}
    \item [\textbf{I.}] Для монотонного векторного поля $f$ в пространстве $\bR^m$ и $n \times m$-матрицы $A$ векторное поле
    \begin{equation*}
        g(x) = Af(A^Tx + a)
    \end{equation*}
    является монотонным полем в пространстве $\bR^n$. Кроме того, если $f$ монотонно с модулем монотонности $\varkappa \geqslant 0$ на замкнутом
    выпуклом множестве $Z \subset \bR^m$, а замкнутое выпуклое множество $X \subset \bR^n$ таково, что 
    $A^T x + a \in Z$ при $x\in X$, то на множестве $X$ поле $g$ монотонно с модулем монотонности $\sigma^2\varkappa$, где $\sigma$ --- минимальное
    сингулярное число матрицы $A$.
    
    \item [\textbf{II.}] Если $S$ --- польское пространство и $f(x, s): \bR^m \times S \to \bR^m$ --- борелевская функция, являющаяся монотонным полем при любом фиксированном $s \in S$, а $\mu(ds)$ --- борелевская вероятностная мера на $S$, а векторное поле
    \begin{equation}
        \label{vec_field_summ}
        F(x) = \int_S f(x,s) \mu(ds)
    \end{equation}
    корректно определено при всех $x$, то поле $F$ монотонно. Кроме того, если $X$ --- замкнутое выпуклое подмножество $\bR^m$, а борелевская функция $\varkappa(s)$ такова, что при всех $s\in S$ поле $f(\cdot, s)$ монотонно на $X$ с модулем монотонности $\varkappa(s)$, то поле $F$ также монотонна на $X$, с модулем монотонности $\int_S \varkappa(s) \mu(ds)$.
    
\end{enumerate}


Далее, пусть $\cX{} \subset \bR^m$ - непустой выпуклый компакт, и $\cH$ -определённое на нём монотонное векторное поле. Вектор $z_*$ называется \emph{слабым решением} вариационного неравенства $\VI(\cX{}, \cH)$, ассоциированного с  если выполнено: 
\begin{equation}
    \label{v_i_weak}
    \cH(z)^T(z - z_*) \ge 0, ~ \forall{x \in \cX{}} 
\end{equation}

Хорошо известно, что слабое решение всегда существует и при этом корень векторного поля им будет. Также в условиях непрерывности $\cH$ будет верно, что слабое решение будет также и сильным решением, то есть:
\begin{equation}
    \label{v_i_strong}
    \cH(z_*)^T(z - z_*) \ge 0, ~ \forall{x \in \cX{}} 
\end{equation}
Также используется следующий факт:
\begin{lemma}
\label{very_good_lemma}
    Если в условиях определения $\VI(\cX{}, \cH)$ поле $\cH$ строго монотонно с модулем монотонности $\varkappa \ge 0$, то слабое решение $z_*$ единственно и выполнено:
    \begin{equation}
        \label{v_i_lemma}
        \cH(z)^T(z - z_*) \ge \varkappa||z - z_*||_2^2
    \end{equation}
\end{lemma}

\subsection{Допускаемые предположения:}
Авторы решают задачу в следующих предположениях:

\begin{itemize}
    \item \textbf{A.1.} Векторное поле $f$ непрерывно и монотонно, а векторное поле
    \begin{equation}
        \label{assumpt_1}
        F(z) = \bE_{\eta \sim Q} \{\eta f(\eta^T z) \}
    \end{equation}
    корректно определено.
    
    \item \textbf{A.2.} $\cX$ --- непустое выпуклое компактное множество, а векторное поле $F$ монотонно и его модуль монотонности $\varkappa$ положителен.
    
    \item \textbf{A.3.} Для некоторого $M < \infty$ и любого $x \in \cX$ выполнено, что
    \begin{equation}
        \label{assumpt_3}
        \bE_{\left(\eta, y\right) \sim P_x} \{\|\eta y\|_2^2 \} < M^2
    \end{equation}
    
    Эти предположения выполнены, в частности, если все моменты случайной величины $\eta$ конечны, а $\bE_{\eta \sim Q} \{\eta \eta^T\} \succ \mathbf{0}$, а на поле $f$ наложены следующие условия:
    \begin{enumerate}
        \item $f$ непрерывно дифференцируема
        \item $d^T f' d > 0$ ля всех $z$ и всех $d \neq 0$ 
        \item $\|f(z)\|_2 \leqslant C(1 + \|z\|_2^p)$ для некоторых $C \geqslant 0,\ p \geqslant 0$ и всех $z$
    \end{enumerate}
\end{itemize}
\section{Основная идея:}
Основная идея происходящего процесса заключается в том, что искомое $x \in \cX$ является корнем следующего векторного поля:
\begin{equation}
    \label{gz_def}
    G(z) = F(z) - F(x),
\end{equation}
где $F$ определено выше. Тогда из предположений \textbf{A.1-3} будет следовать, что $G$ строго монотонно на $\cX$, из чего будет следовать что искомый корень единственен. Далее можно заметить,что корень является слабым решением  $\VI(G_{\omega^K})$. 
, $$
    z_{k}=\Proj_{\cX}[z_{k-1}-\gamma_k G(z_{k-1})],\,k=1,2,...,K,
$$
где
\begin{itemize}
    \item $\Proj_{\cX}[z]=\argmin_{u\in\cX} \|z-u\|_2;$
    \item $\gamma_k>0$ данный размер шага;
    \item начальная $z_0$ выбирается произвольно из $\cX$.
\end{itemize}
Известно, что в предположениях \textbf{A.1-3}, этот метод дает сколь угодно хорошее приближение решения
для достаточно большого $K$. Трудность заключается в том, что значения $G$ неизвестны, а есть лишь реализации случайных величин. Поэтому необходимо ввести аналогичные векторные поля для наших наблюдений:
\begin{equation}
    \label{g_en_def}
    G_{\eta,y}(z) =\eta f(\eta^Tz)- \eta y:
\end{equation}

\begin{lemma}
\label{good_lemma}
    $\forall{x} \in \cX{}$ верно:
    \begin{enumerate}
        \item  $\bE_{(\eta,y) \sim P_x} \{ G_{\eta,y}(z)\} = G(z)~ \forall{z \in \bR^n}$
        \item $ ||F(x)||_2 \leq M$
        \item  $\bE_{(\eta,y) \sim P_x} \{ ||G_{\eta,y}(z)||_2^2\} \leq 4M^2 ~ \forall{z \in \cX{}}$
    \end{enumerate}
\end{lemma}

Данная теория даёт возможность пользоваться двумя стандартными подходами.

\subsection{Аппроксимация средними}
Лемма \eqref{good_lemma} говорит о том, что $G_{\eta,y}(z)$ есть несмещённая оценка $G(z)$ с равномерно ограниченными по $y,z$ конечными матожиданиями и дисперсией, поэтому поле \begin{equation}
    \label{g_w}
    G_{\omega^K}(z) =\frac{1}{K} \sum\limits_{k=1}^K[\eta_k f(\eta_k^Tz)- \eta_k y_k]
\end{equation}
равномерно сходится по распределению к $G(z)$ вследствие закона больших чисел. Тогда из строгой монотонности $f$ и леммы \eqref{very_good_lemma} будет следовать строгая монотонность $G_{\omega^K}(z)$, откуда в свою очередь следует то, что асимптотически почти наверное слабое решение $\VI{}(G_{\omega^K}, \cX)$ сходится к слабому решению $\VI{}(G, \cX)$, а предпоследнее уже можно вычислить эффективно.   

\subsection{Стохастическая аппроксимация}
Также можно воспользоваться стохастическим приближением, имея в виду несмещенность оценки $G_{(\eta_k,y_k)}(z)$ для $G(z)$. Тогда получим следующую реккуренту:
\begin{equation}\label{logiteqSA}
z_{k}=\Proj_{\cX}[z_{k-1}-\gamma_k G_{(\eta_k,y_k)}(z_{k-1})],\,1\leq k\leq K,
\end{equation}
Для нее верно следующее
\begin{proposition}
    В предположении выполнения \textbf{A.1-3}, с шагами: 
   \begin{equation}\label{logiteq50}
        \gamma_k=[\varkappa (k+1)]^{-1},\,k=1,2,...
    \end{equation} 
    Для любого сигнала $x \in \cX$ последовательность оценок $\widehat{x}_k(\omega^{k})=z_{k}$, полученная из рекурренты {\rm (\ref{logiteqSA}})
    при $\omega_k=(\eta_k,y_k)$ для каждого $k$ подчиняется закону:
    \begin{equation}\label{logiteq51}
        \bE_{\omega^k\sim P_x^k}\left\{\|\widehat{x}_k(\omega^k)-x\|_2^2\right\}\leq {4M^2\over\varkappa^2(k+1)},\,k=0,1,...
    \end{equation}
\end{proposition}
Из этого утверждения мы можем сделать вывод, что просто реализовав данную рекурренту мы получаем хорошее приближение для решения.
\end{document}

